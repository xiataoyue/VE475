\documentclass[12pt, a4paper]{article}
\usepackage[UTF8]{ctex}
\usepackage{enumerate}
\usepackage{amsmath}
\usepackage{amssymb}
\usepackage{blkarray}

\begin{document}
\title{VE475 Intro to Cryptography Homework 3}
\author{Taoyue Xia, 518370910087}
\date{2021/06/02}
\maketitle

\section{Ex1}
\begin{enumerate}
    \item Take X's value as 0, 1, 2 in $\mathbb{F}_3[X]$:
          $$0^2 + 1 = 1\ mod\ 3 \quad 1^2 + 1 = 2\ mod\ 3 \quad 2^2 + 1 = 2\ mod\ 3$$
          We can see that for $X \in \mathbb{F}_3[X]$, there doesn't exists an $X$ which makes $X^2 + 1 = 0\ mod\ 3$\newline
          Thus $X^2 + 1$ is irreducible in $\mathbb{F}_3[X]$.
    \item In question 1, we proved that $X^2 + 1$ is irreducible in $\mathbb{F}_3[X]$, 
          and the polynomial $1 + 2X$ 's degree is less than 2, according to the proof on page 39, c2, 
          Let $P(X) = X^2 + 1$, $A(X) = 1 + 2X$, then there always exists a $B(X)$, such that
          $A(X)\, B(X) = 1\ mod\ P(X)$, which means $B(x)$ is the multiplication inverse of $1 + 2X\ mod\ X^2 + 1$. 
          Proof done.
    \item Apply the extended Euclidean algorithm, let $a$ and $b$ be such that $a(1 + 2X) + b(X^2 + 1) = 1\ mod\ 3$.
          Then calculate in matrix form(a's value in the first column, b's value in the second):\newline
          $$
          \begin{pmatrix} 1 & 0 & 1 + 2X\\ 0 & 1 & X^2 + 1 \end{pmatrix}
          \Rightarrow \begin{pmatrix} 0 & 1 & X^2 + 1\\ 1 & 0 & 1 + 2X \end{pmatrix}
          \Rightarrow \begin{pmatrix} 1 & 0 & 1 + 2X\\ X & 1 & X + 1 \end{pmatrix}
          $$
          $$
          \Rightarrow \begin{pmatrix} X & 1 & X + 1\\ X + 1 & 1 & 2 \end{pmatrix}
          \Rightarrow \begin{pmatrix} X + 1 & 1 & 2\\ X^2 + 2X & X + 1 & 1 \end{pmatrix}
          $$
          Thus we can find that the multiplication inverse of $1 + 2X\ mod\ X^2 + 1$ is $X^2 + 2X$.

\end{enumerate}

\section{Ex2}
\begin{enumerate}
    \item The \emph{InvShiftRows} function cyclicly shift each row \emph{i}'s elements right for $i = 0,\ 1,\ 2\, 3$.\newline
          For example, if the $4\times 4$ matrix is 
          $\begin{bmatrix} a_{00} & a_{01} & a_{02} & a_{03}\\
            a_{11} & a_{12} & a_{13} & a_{10}\\
            a_{22} & a_{23} & a_{20} & a_{21}\\
            a_{33} & a_{30} & a_{31} & a_{32}
            \end{bmatrix}$, then the matrix after the operation \emph{InvShiftRow} would be: 
          $\begin{bmatrix} a_{00} & a_{01} & a_{02} & a_{03}\\
            a_{10} & a_{11} & a_{12} & a_{13}\\
            a_{20} & a_{21} & a_{22} & a_{23}\\
            a_{30} & a_{31} & a_{32} & a_{33}
           \end{bmatrix}$.
\end{enumerate}

\end{document}