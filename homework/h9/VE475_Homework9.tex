\documentclass[12pt, a4paper]{article}
\usepackage[UTF8]{ctex}
\usepackage{enumerate}
\usepackage{amsmath}
\usepackage{amssymb}
\usepackage{blkarray}
\usepackage{geometry}

\geometry{left = 2.0cm, right = 2.0cm}

\begin{document}
\title{VE475 Intro to Cryptography Homework 8}
\author{Taoyue Xia, 518370910087}
\date{2021/07/15}
\maketitle

\section*{Ex1 --- Missile or not missile}
We can simply use a $(t,w)$-\textit{threshold scheme}, or to be more specific, the Shamir threshold scheme, 
to solve this problem. First we can calculate $lcm(1,2,5) = 10$, so we can divide the secret into 10 shares. 
The general has 10 shares, each colonel has 5 shares and each desk clerk has 2 shares. 
For the previous setting, the problem is solved.

\section*{Ex2 --- Asmuth-Bloom Threshold Secret Sharing Scheme}
The Asmuth-Bloom Threshold Secret Sharing Scheme uses the Chineses Remainder Theorem to determine a secret based on multiple modulos.\\
We consider a sequence of pairwise coprime positive integers $m_0 < \cdots < m_n$, 
with $2 \leq k \leq n$ be an integer, and $m_0\cdot m_{n-k+2}\cdots m_n < m_1\cdots m_k$. For this sequence, 
we can choose the secret S in the set $\mathbb{Z}/m_0\mathbb{Z}$.\\
We then pick a random integer $\alpha$ such that $S + \alpha\cdot m_0 < m_1\cdots m_k$. 
We will compute the reduction modulo $m_i$ of $S + \alpha\cdot m_0$, for all $1 \leq i \leq n$, as $s_i$, 
then these are the shares $I_i = (s_i, m_i)$. 
Now for any $k$ different shares $I_{i_1},\cdots,I_{i_k}$, we consider the system of congruences:
$$
\begin{cases}
    x \equiv s_{i_1}\ mod\ m_{i_1}\\
    \qquad \vdots\\
    x \equiv s_{i_k}\ mod\ m_{i_k}
\end{cases}
$$
Then according to the Chinese Remainder Theorem, since all $m_i$ are pairwise coprime, 
the system will have a unique solution $S_0$ modulo $m_{i_1}\cdots m_{i_k}$.\\
We can say that the Asmuth-Bloom scheme is perfect since $\alpha$ is independent of $S$ and
$$
\left.{\begin{array}{r}\prod _{{i=n-k+2}}^{n}m_{i}\\\alpha \end{array}}\right\}<{\frac  {\prod _{{i=1}}^{k}m_{i}}{m_{0}}}
$$
Therefore, $\alpha$ can be any integer modulo
$$\prod_{i=n-k+2}^n m_i$$
This product of $k − 1$ moduli is the largest of any of the $n$ choose $k − 1$ possible products, 
therefore any subset of $k − 1$ equivalences can be any integer modulo its product, and no information from $S$ is leaked.

\section*{Ex3 --- Shamir’s Threshold Secret Sharing Scheme}
The Lagrange Interpolation method can be expressed as:
$$
\ell_i(x) = \prod_{\begin{smallmatrix} 0\leq m\leq k\\ m \neq i \end{smallmatrix}} \frac{x - x_m}{x_i - x_m} = 
\frac{(x - x_0)\cdots (x - x_{i-1})(x - x_{i+1})\cdots (x - x_k)}{(x_i - x_0)\cdots (x_i - x_{i-1})(x_i - x_{i+1})\cdots (x_i - x_k)}
$$
$$L(x) = \sum_{i=0}^k y_i \ell_i(x)$$

Using the example on slide 6.10, we set $p = 1234567890133$, $m = 190503180520$, $r_1 = 482943028839$, 
and $r_2 = 1206749628665$. We can first construct $S(X)$ as:
$$S(X) = 190503180520 + 482943028839X + 1206749628665X^2$$
We take three values $x_i$ and corresponding $S(x_i)$ as $y_i$:
$$x_0 = 2,\quad y_0 = 1045116192326$$
$$x_1 = 3,\quad y_1 = 154400023692$$
$$x_2 = 7,\quad y_2 = 973441680328$$
Then we can calculate each $\ell_i(x)$ as follows:
$$\ell_0(x) = \frac{(x-3)(x-7)}{(2-3)(2-7)} = \frac{1}{5}(x-3)(x-7)$$
$$\ell_1(x) = \frac{(x-2)(x-7)}{(3-2)(3-7)} = -\frac{1}{4}(x-2)(x-7)$$
$$\ell_2(x) = \frac{(x-2)(x-3)}{(7-2)(7-3)} = \frac{1}{20}(x-2)(x-3)$$
After combining, the final polynomial is:
\begin{align*}
    L(x) &= \sum_{i=0}^2 y_i\ell_i(x)\\
         &= \frac{1045116192326}{5}(x-3)(x-7) - \frac{154400023692}{4}(x-2)(x-7) + \frac{973441680328}{20}(x-2)(x-3)\\
         &= \frac{1095476582793}{5}x^2 - 1986192751427x + \frac{20705602144728}{5}
\end{align*}
Using the Extended Euclidean algorithm, we can easily find the inverse of $5$ modulo p as $740740734080$. 
Then we can determine that:
$$r_2 \equiv \frac{1095476582793}{5} \equiv 1095476582793\cdot 740740734080 \equiv 1206749628665\ mod\ p$$
$$r_1 \equiv -1986192751427 \equiv 482943028839\ mod\ p$$
$$m \equiv \frac{20705602144728}{5} \equiv 20705602144728\cdot 740740734080 \equiv 190503180520\ mod\ p$$
In this way, we can recover the secret message.

\section*{Ex4 --- Simple questions}
\begin{enumerate}
    \item From the two plains from Alice and Bob, we know that:
          $$z = 2x + 3y + 13 = 5x + 3y + 1 \quad \Rightarrow \quad x = 4,\ z = 3y + 21$$
          where $x = 4$ is the secret. Therefore, Alice and Bob don't need the help of Charly.
    \item We know the fact that if one adds to a column of a matrix the product by a scalar of another column, 
          then the determinant remains unchanged. So we can subtract each column with the previous column multiplied by $x_1$, 
          starting from the rightmost colum, except for the first column, then it gives us the following matrix:
          $$
          \begin{bmatrix}
              1 & 0 & 0 & \cdots & 0\\
              1 & x_2 - x_1 & x_2(x_2 - x_1) & \cdots & x_2^{t-2}(x_2 - x_1)\\
              1 & x_3 - x_1 & x_3(x_3 - x_1) & \cdots & x_3^{t-2}(x_3 - x_1)\\
              \vdots & \vdots & \vdots & \ddots & \vdots\\
              1 & x_t - x_1 & x_t(x_t - x_1) & \cdots & x_t^{t-2}(x_t - x_1)
          \end{bmatrix}
          $$
          Denote the Vandermonde matrix as $V$, the above matrix as $W$, we know that $\det(V) = \det(W)$. 
          According to the Laplace expansion formula along the first row, we can transform matrix $W$ into:
          $$L = 
          \begin{bmatrix}
              x_2 - x_1 & x_2(x_2 - x_1) & \cdots & x_2^{t-2}(x_2 - x_1)\\
              x_3 - x_1 & x_3(x_3 - x_1) & \cdots & x_3^{t-2}(x_3 - x_1)\\
              \vdots & \vdots & \ddots & \vdots\\
              x_t - x_1 & x_t(x_t - x_1) & \cdots & x_t^{t-2}(x_t - x_1)
          \end{bmatrix}
          $$
          And we can obviously find out that $\det(L) = \det(W) = \det(V)$. 
          Then we notice that each row has a factor $x_k - x_1,\ k\in \{2,3,\cdots,t\}$. 
          So we can express $\det(L)$ as follows:
          $$
          \det(L) = \prod_{1<k\leq n} (x_k - x_1) 
          \begin{vmatrix}
              1 & x_2 & x_2^2 & \cdots & x_2^{t-2}\\
              1 & x_3 & x_3^2 & \cdots & x_3^{t-2}\\
              \vdots & \vdots & \vdots & \ddots & \vdots\\
              1 & x_t & x_t^2 & \cdots & x_t^{t-2}
          \end{vmatrix} = \det(V^\prime)
          $$
          where $V^\prime$ is a Vandermonde matrix from $x_2$ to $x_t$. 
          So we iterate the process until 
          $$\det(V^\prime) = 
          \begin{vmatrix}
              1 & x_{t-1}\\
              1 & x_t
          \end{vmatrix} = x_t - x_{t-1}
          $$
          Therefore, we can conclude that:
          \begin{align*}
              \det(V) &= (x_t-x_1)\cdots(x_t-x_{t-1})(x_{t-1}-x_1)\cdots(x_{t-1}-x_{t-2})\cdots\cdots (x_2-x_1)\\
                      &= \prod_{1\leq j < k\leq t}(x_k - x_j)
          \end{align*}
          Therefore, the proof is done.
    \item The evaluation website for VE475 has not opened so far.
          
\end{enumerate}

\section*{Ex5 --- Reed Solomon codes}
\begin{enumerate}
    \item The Reed–Solomon code is actually a family of codes, where every code is characterised by three parameters: 
          an alphabet size $q$, a block length $n$, and a message length $k$, with $k < n \leq q$. 
          The set of alphabet symbols is interpreted as the finite field of order $q$, thus $q$ has to be a prime power. 
          the block length is usually some constant multiple of the message length, that is, the rate $R = k/n$ is some constant, 
          and the block length is equal to or one less than the alphabet size, that is, $n = q$ or $n = q - 1$.\\
          Every codeword of the Reed–Solomon code is a sequence of function values of a polynomial of degree less than k. 
          The message symbols are treated as the coefficients a polynomial $p$ of degree less than $k$, 
          over the finite field $\mathbb{F}$ with $q$ elements. In turn, 
          the polynomial $p$ is evaluated at $n \leq q$ distinct points $\{a_1,\dots, a_n\}$ of the field $\mathbb{F}$, 
          and the sequence of values is the corresponding codeword.\\
          Formally, the set $\mathcal{C}$  of codewords of the Reed–Solomon code is defined as follows:
          $$\mathcal{C} = \{(p(a_1),\dots,p(a_n))\, | \, p\ \text{is a polynomial over }\mathbb{F}\ \text{of degree} < k\}$$
    \item Since any two distinct polynomials of degree less than $k$ agree in at most $k-1$ points, 
          this means that any two codewords of the Reed–Solomon code disagree in at least $n-(k-1)=n-k+1$ positions. 
          So the distance of the Reed–Solomon code is exactly $D=n-k+1$.\\
          According to theorem 7.16, if $D > k(1- 1/w^2)$, where $w$ is the size of coalition, 
          and $k$ is the code length in Reed-Solomon code, then it is possible to identify a parent of descendant of $\mathcal{C}$. 
          Therefore, for $w = 2$, we can obtain that
          \begin{align*}
              D = n - k + 1 &> k(1 - \frac{1}{w^2})\\
              k &< \frac{4}{7}(n+1)
          \end{align*}
          So we can conclude that k should satisfy the condition $k < \frac{4}{7}(n + 1)$. 
          
\end{enumerate}




\end{document}