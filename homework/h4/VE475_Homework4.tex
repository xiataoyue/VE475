\documentclass[12pt, a4paper]{article}
\usepackage[UTF8]{ctex}
\usepackage{enumerate}
\usepackage{amsmath}
\usepackage{amssymb}
\usepackage{blkarray}
\usepackage{geometry}
\geometry{left = 2.0cm, right = 2.0cm}

\begin{document}
\title{VE475 Intro to Cryptography Homework 4}
\author{Taoyue Xia, 518370910087}
\date{2021/06/16}
\maketitle

\section*{Ex1}
\begin{enumerate}
    \item Let $G$ be a group, and G = $\mathbb{Z}/p^k\mathbb{Z}$, thus it has $p^k - 1$ elements, which are $1, 2,\cdots, p^k - 1$.
          As $p$ is a prime number, the numbers in $G$ which are not coprime with $p^k$ are $p, 2p, \cdots, (p-1)p^{k-1}$, 
          which have $p^k / p = p^{k-1}$ in total.\newline
          Thus according to Euler's totient function, $\varphi(p^k) = p^k - p^{k-1} = p^{k-1}(p - 1)$
    \item Suppose that $\varphi(m) = m - k_m$, where $k_m$ denotes the number of elements which are not coprime with $m$ in $\mathbb{Z}/m\mathbb{Z}$.
          Similarly, $\varphi(n) = n - k_n$. For $\mathbb{Z}/mn\mathbb{Z}$, as $m$ and $n$ are coprime integers, 
          $m$ and $n$ has no common divisor, 
          so the elements that are not coprime with $mn$ in the ring are the combination of those in ring $M$ and $N$, 
          adding the multiplication between each of them in $M$ and $N$. Thus:
          $$\varphi(mn) = mn - k_m - k_n - k_m\cdot k_n = (m - k_m)(n - k_n) = \varphi(m)\varphi(n)$$
          Proof done.
    \item Suppose that $n = p_1^{k_1}\cdot p_2^{k_2}\cdots p_n^{k_n}$, where $p_1\cdots p_n$ are different prime integers 
          and $k_1\cdots k_n \geq 1$ are integers. We can see that for all $p_i^{k_i},\ i = 1, \cdots, n$, 
          they are coprime with one another, thus:
          \begin{align*}
              \varphi(n) &= \varphi(p_1^{k_1})\varphi(p_2^{k_2})\cdots \varphi(p_n^{k_n})\\
                         &= p_1^{k_1 - 1}(p_1 - 1)\cdot p_2^{k_2 - 1}(p_2 - 1) \cdots p_n^{k_n - 1}(p_n - 1)\\
                         &= p_1^{k_1}(1 - \frac{1}{p_1})\cdot p_2^{k_2}(1 - \frac{1}{p_2})\cdots p_n^{k_n}(1 - \frac{1}{p_n})\\
                         &= p_1^{k_1}p_2^{k_2}\cdots p_n^{k_n} \cdot (1 - \frac{1}{p_1})(1 - \frac{1}{p_2})\cdots (1 - \frac{1}{p_n})\\
                         &= n \prod_{p|n}(1 - \frac{1}{p})
          \end{align*}
          Proof done.
    \item To calculate $7^{803}$'s last three digits, we can calculate $7^{803}\ mod\ 1000$.\newline
          First, use Euler's totient function and the above equations to calculate $\varphi(1000)$. 
          We can easily know that 1000 has two prime divisors 2 and 5, thus using the equation in (3):
          $$\varphi(1000) = 1000\cdot (1 - \frac{1}{2})\cdot (1 - \frac{1}{5}) = 400$$
          As 7 and 1000 are coprime, using Euler's Theorem, we can get:
          $$7^{\varphi(1000)} = 7^{400} = 1\ mod\ 1000$$
          Finally, we can calculate $7^{803}\ mod\ 1000$ as:
          \begin{align*}
              7^{803} &= (7^{400})^2\cdot 7^3\ mod\ 1000\\
                      &= 343\ mod\ 1000
          \end{align*}
          Therefore, we can conclude that the last three digits of $7^{803}$ is 343.
          
\end{enumerate}

\section*{Ex2}
\begin{enumerate}
    \item For round 1, the original key is used, thus the key used is 128 bits of 1.
    \item $$K(5) = K(4) \oplus K(1)$$
    \item It is obvious that for a 4-bit number $X$:
          $$X \oplus 1111 = \overline{X}$$
          As the key of first round is 128-bits 1, thus:
          $$K(0) = K(1) = K(2) = K(3) = \begin{pmatrix} 1111\\ 1111\\ 1111\\ 1111\end{pmatrix}$$
          Then we can calculate $K(10)$ as:
          \begin{align*}
              K(10) &= K(9) \oplus K(6)\\
                    &= (K(8) \oplus K(5)) \oplus (K(5) \oplus K(2))\\
                    &= K(8) \oplus K(2)\\
                    &= \overline{K(8)}
          \end{align*}
          For $K(11)$:
          \begin{align*}
            K(11) &= K(10) \oplus K(7)\\
                  &= (K(9) \oplus K(6)) \oplus (K(6) \oplus K(3))\\
                  &= K(9) \oplus K(3)\\
                  &= \overline{K(9)}
        \end{align*}
        Proof done.
\end{enumerate}

\section*{Ex3}
\begin{enumerate}
    \item For ECB mode, every block of plaintext is seperately encrypted by a transform $E_K$, 
          which takes $E$ as a transform function, and $K$ as a key. 
          So we know that the corruption of one block does not have influence on other blocks. 
          Therefore, the number of plaintext encrypted incorrectly for ECB mode is one.\newline
          For CBC mode, starting from the second block, every block will perform an ``xor" transformation 
          with the previous encrypted block. In this sense, if the corrupted block is not the last one, 
          the corrupted block and the next block will be incorrectly encrypted. Therefore, 
          the number of plaintext encrypted incorrectly for CBC mode is two.
    \item For a chosen plaintext $P$, The encryption function $E$, and the initial $IV_0$, 
          we can have the ciphertext $C$ as:
          $$C = E(IV_0 \oplus P) = E(IV_1 \oplus (IV_1 \oplus IV_0 \oplus P))$$
          As $IV$ increments by 1 each time, it will reset after reaching max bits. 
          So after one round of $IV$, we can easily know the exact composition of each $IV$, 
          thus it is easier to deduce the encryption function and the key.\newline
          So it is not CPA secure under this circumstance.
    \item The order of $U(\mathbb{Z}/29\mathbb{Z})$ is 28, then we calculate $2^i\ mod\ 29$ in the following table:
          \begin{center}
              \begin{tabular}{cccccccc}
                \hline
                $i$ & $2^i\ mod\ 29$ & $i$ & $2^i\ mod\ 29$ & $i$ & $2^i\ mod\ 29$ & $i$ & $2^i\ mod\ 29$\\
                \hline
                1 & 2 & 8 & 24 & 15 & 27 & 22 & 5\\
                2 & 4 & 9 & 19 & 16 & 25 & 23 & 10\\
                3 & 8 & 10 & 9 & 17 & 21 & 24 & 20\\
                4 & 16 & 11 & 18 & 18 & 13 & 25 & 11\\
                5 & 3 & 12 & 7 & 19 & 26 & 26 & 22\\
                6 & 6 & 13 & 14 & 20 & 23 & 27 & 15\\
                7 & 12 & 14 & 28 & 21 & 17 & 28 & 1\\
                \hline
              \end{tabular}
          \end{center}
          From the table we can easily see that $2^i,\ (i = 1, 2, \cdots, 28)$ generates all the elements 
          in $U(\mathbb{Z}/29\mathbb{Z})$. So 2 is a generator of $U(\mathbb{Z}/29\mathbb{Z})$.
          \vspace{0.3cm}
          
          Or we can use the method introduced in c3, page 17.\newline
          First, $p = 29$ is a prime integer, $2 \in U(\mathbb{Z}/29\mathbb{Z})$, 
          and $p - 1 = 28$ have two prime divisors: $q_1 = 2$ and $q_2 = 7$. Thus we calculate:
          $$2^{\frac{p-1}{q_1}} = 2^{14} \equiv 28\ mod\ 29$$
          $$2^{\frac{p-1}{q_2}} = 2^4 \equiv 16\ mod\ 29$$
          As $2^{(p-1)/q} \not\equiv 1\ mod\ 29$, thus 2 is a generator of $U(\mathbb{Z}/29\mathbb{Z})$.
    \item As 1801 and 8191 are two prime numbers, according to the lagrange symbol's definition, 
          we just need to calculate $1801^{\frac{8191-1}{2}} = 1801^{4095}\ mod\ 8191$.\newline
          Apply the Modular exponentiation method, and with the code in ``ex3\_4.cpp", 
          we can finally calculate $1801^{4095} \equiv 8190 \equiv -1\ mod\ 8191$. 
          Thus we can conclude that $(\frac{1801}{8191}) = -1$.
    \item For the equation $ax^2 + bx + c = 0$, it has two roots $x = \frac{-b \pm \sqrt{b^2 - 4ac}}{2a}$. \newline
          When $b^2 -4ac = 0$, which means the equation has one root $x = -\frac{b}{2a}$, 
          then $(\frac{b^2 - 4ac}{p}) = 0$, the equation holds.\newline
          When $b^2 - 4ac \neq 0$, Then:
          $$\frac{-b \pm \sqrt{b^2 - 4ac}}{2a} = x\ mod\ p$$
          $$(b^2 - 4ac)^{\frac{1}{2}} = (2ax \pm b)\ mod\ p$$
          As $p$ is an odd prime, and $a \not\equiv 0\ mod\ p$, we can show that:
          $$(b^2 - 4ac)^{\frac{p-1}{2}} = (2ax \pm b)^{p-1}\ mod\ p$$
          Then if $2ax \pm b \not\equiv 0\ mod\ p$, according to the Fermat's little theorem, 
          we can find $(b^2 - 4ac)^{\frac{p-1}{2}} \equiv 1\ mod\ p$, thus $b^2 - 4ac$ is a square mod p, 
          which means that $(\frac{b^2 - 4ac}{p}) = 1$. Then the equation has two roots. 
          Otherwise, $(\frac{b^2 - 4ac}{p}) = -1$, the equation has no root.\newline
          In all, we prove that the number of solutions mod $p$ to the equation $ax^2 + bx + c = 0$ 
          is $1 + (\frac{b^2 - 4ac}{p})$.
    \item As $p$ and $q$ are two primes, we can have:
          $$n^{p-1} \equiv 1\ mod\ p \eqno{(1)}$$
          $$n^{q-1} \equiv 1\ mod\ p$$
          Since $q-1$ divides $p-1$, there exists a positive integer $k$, so that $p-1 = k(q-1)$, then:
          $$(n^{q-1})^k = n^{p-1} \equiv 1\ mod\ q \eqno{(2)}$$
          Finally, because $gcd(n,pq) = 1$, which means $n$ and $pq$ are coprime, 
          applying the CRT to equation (1) and (2), we can get:
          $$n^{p-1} \equiv 1\ mod\ pq$$
          Proof done.
    \item Proof
          \begin{enumerate}[($\Rightarrow$)]
              \item For $(\frac{-3}{p}) = 1$, as $p$ is an odd prime, we can decomposite the former lagrange symbol as:
                    $$(\frac{-3}{p}) = (\frac{-1}{p})(\frac{3}{p}) = 1$$
                    Firstly, if $(\frac{-1}{p}) = 1$, then $(-1)^{\frac{p-1}{2}} \equiv 1\ mod\ p$, 
                    which indicates that $p \equiv 1\ mod\ 4$. Then $(\frac{3}{p}) = 1$. 
                    As $p \not\equiv 3\ mod\ 4$, according to Jacobi symbol, 
                    $(\frac{3}{p}) = (\frac{p}{3}) = 1$, thus 
                    $p^{\frac{3-1}{2}} = p \equiv 1\ mod\ 3$. \newline
                    Then, if $(\frac{-1}{p}) = -1$, then similarly we can have 
                    $p \equiv 3\ mod\ 4$. This means $(\frac{3}{p}) = -1$. According to the Jacobi symbol, 
                    $(\frac{3}{p}) = -(\frac{p}{3}) = -1\ \Rightarrow \ (\frac{p}{3}) = 1$, 
                    which means that $p \equiv 1\ mod\ 3$.\newline
                    According to both of the conditions, we can prove that 
                    if $(\frac{-3}{p}) = 1$, then $p \equiv 1\ mod\ 3$ 
          \end{enumerate}
          \begin{enumerate}[($\Leftarrow$)]
              \item Since we know that $p \equiv 1\ mod\ 3$, we can have $(\frac{p}{3}) = 1$.\newline
                    As $p$ is an odd prime, $p \equiv 1\ mod\ 4$ or $p \equiv 3\ mod\ 4$.\newline
                    When $p \equiv 1\ mod\ 4$, we can get $(\frac{3}{p}) = (\frac{p}{3}) = 1$, 
                    and $(\frac{-1}{p}) = 1$, thus:
                    $$(\frac{-3}{p}) = (\frac{3}{p})(\frac{-1}{p}) = 1\cdot 1 = 1$$
                    When $p \equiv 3\ mod\ 4$, we can get $(\frac{3}{p}) = -(\frac{p}{3}) = -1$, 
                    and $(\frac{-1}{p}) = -1$, thus:
                    $$(\frac{-3}{p}) = (\frac{3}{p})(\frac{-1}{p}) = (-1)\cdot (-1) = 1$$
                    According to the above procedure, we can prove that 
                    if $p \equiv 1\ mod\ 3$, $(\frac{-3}{p}) = 1$.
          \end{enumerate}
    \item We don't take $p = 2$ into account as $1^{\frac{1}{2}} = 1\ mod\ 2$, it does not representitive.\newline
          Then $p$ is an odd prime, and 2 is a factor of the order of $\mathbb{F}^{*}_p$.\newline
          Since $(\frac{a}{p}) = 1$, it is for sure that:
          $$a^{\frac{p-1}{2}} \equiv 1\ mod\ p$$
          However, if $a$ is a generator of $\mathbb{F}^{*}_p$, for all primes $q$ that $q|(p-1)$, 
          $$a^{\frac{p-1}{q}} \not\equiv 1\ mod\ p$$
          This raises a contradiction with the property deduced above.\newline
          Therefore, it is proved that if $(\frac{a}{p}) = 1$, then $a$ is not a generator of $\mathbb{F}^{*}_p$. 

\end{enumerate}

\section*{Ex4}
\begin{enumerate}
    \item We will prove by contradiction.\newline
          Suppose that there exists a prime $p$ that is irreducible in the integral domain, 
          which means that $p = mn$, for some $m$, $n$ that are non-zero, non-invertible elements. 
          According to property (*), there exists some $k_1, k_2 \neq 0$ so that $p|(k_1k_2mn)$. 
          Let $x = k_1m$, $y = k_2n$. If $n\nmid k_1$ and $m\nmid k_2$, then $p = mn\nmid k_1m$ 
          and $p = mn\nmid k_2n$, which raises a contradiction to (*).\newline
          Therefore, in an integral domain, any prime element is irreducible.
    \item Still prove by contradiction.\newline
          Suppose that there exists an irreducible element $p$ in $\mathbb{Z}$ but it is not a prime. 
          In this sense, $p$ cannot be represented by the multiplication of two non-zero elements, 
          which indicates that $p \neq mn$ for any $m, n\in \mathbb{Z}$. In other words, 
          $p$'s divisor is only 1 and itself. However, according to (**), 
          this property indicates that $p$ is a prime, which raises a contradiction.\newline
          Therefore, in $\mathbb{Z}$ any irreducible element is a prime in the classical sense (**).
    \item Let $p\in \mathbb{Z}$ be an irreducible element, which is a prime. 
          Suppose there exist $x, y\in \mathbb{Z}$ that $p | (x\cdot y)$. 
          In this sense, $x\cdot y = k\cdot p$, where $k\in \mathbb{Z}$ is some arbitrary integer.\newline
          Firstly, if $k$ is irreducible too, then $x = k$ and $y = p$, which indicates $p | y$.\newline
          If $k$ is not irreducible, then divide $k$ into $k = k_1\cdot k_2$, where $k_1, k_2 \neq 0, 1$, 
          thus $x\cdot y = k_1k_2p$. Then combination of $x$ and $y$ are shown in the table below:
          \begin{center}
                \begin{tabular}{c|c}
                      \hline
                      $x$ & $y$\\
                      \hline
                      $k_1$ & $k_2p$\\
                      $k_2$ & $k_1p$\\
                      $p$ & $k_1k_2$\\
                      $k_1k_2$ & $p$\\
                      $k_1p$ & $k_2$\\
                      $k_2p$ & $k_1$
                \end{tabular}
          \end{center}
          From the above table, we can easily find that for any $k\in \mathbb{Z}$, 
          $p|x$ or $p|y$ is always true.\newline
          Therefore, for $p\in \mathbb{Z}$, (**) implies (*).
    \item We need to prove (*) implies (**), using contradiction.\newline
          As $p$ is an integer, (*) indicates that if $p | (x\cdot y)$, 
          then at least one of $x$ and $y$ would be a multiple of $p$, 
          and $p$ is irreducible.\newline
          Suppose that (**) does not hold taking (*) as the prerequisite. 
          This indicates that there exists some $a$, which satisfies $a\neq 1$, $a\neq p$, 
          but $a | p$. However, this raises a contradiction with (*) that $p$ is irreducible. 
          Therefore, for $p\in \mathbb{Z}$, (*) implies (**).\newline
          Since we have proved in question 3 that (**) implies (*) for $p\in \mathbb{Z}$, 
          we can have the conclusion that (*) and (**) are equivalent for integers.

\end{enumerate}

\section*{Ex5}
\begin{enumerate}
    \item Similar to Ex3.4, this time, apply the "Right-to-left binary method" of Modular exponentiation. 
          Since $65537 = 65536 + 1 = 2^{16} + 1$, we just need to calculate $3^{\frac{65537-1}{2}} = 3^{2^{15}}\ mod\ 65537$, 
          the procedure is shown in the table below:
          \begin{center}
              \begin{tabular}{c|c}
                \hline
                $i_{th}$ multi & modulo 65537\\
                \hline
                $3^{2^0}$ & $1\cdot 3 \equiv 3\ mod\ 65537$\\
                $3^{2^1}$ & $3^2 \equiv 9\ mod\ 65537$\\
                $3^{2^2}$ & $9^2 \equiv 81\ mod\ 65537$\\
                $3^{2^3}$ & $81^2 \equiv 6561\ mod\ 65537$\\
                $3^{2^4}$ & $6561^2 \equiv 54449\ mod\ 65537$\\
                $3^{2^5}$ & $54449^2 \equiv 61869\ mod 65537$\\
                $3^{2^6}$ & $61869^2 \equiv 19139\ mod\ 65537$\\
                $3^{2^7}$ & $19139^2 \equiv 15028\ mod\ 65537$\\
                $3^{2^8}$ & $15028^2 \equiv 282\ mod\ 65537$\\
                $3^{2^9}$ & $282^2 \equiv 13987\ mod\ 65537$\\
                $3^{2^{10}}$ & $13987^2 \equiv 8224\ mod\ 65537$\\
                $3^{2^{11}}$ & $8224^2 \equiv 65529\ mod\ 65537$\\
                $3^{2^{12}}$ & $65529^2 \equiv 64\ mod\ 65537$\\
                $3^{2^{13}}$ & $64^2 \equiv 4096\ mod\ 65537$\\
                $3^{2^{14}}$ & $4096^2 \equiv 65281\ mod\ 65537$\\
                $3^{2^{15}}$ & $65281^2 \equiv 65536\ mod\ 65537$\\
              \end{tabular}
          \end{center}
          Since $3^{32768} \equiv 65536\ mod\ 65537 \not\equiv 1\ mod\ 65537$, 
          we can get the result that $(\frac{3}{65537}) = -1$
    \item From question 1, we obtain that $3^{32768} \equiv 65536\ mod\ 65537$, 
          thus it is obvious that $3^{32768} \not\equiv 1\ mod\ 65537$.
    \item Firstly, we find that for $p = 65537$, $p-1 = 65536$, which has only one prime divisor 2. 
          Then, according to the theorem on page 17, c3, 
          since $3^{\frac{65537-1}{2}} = 3^{32768} \not\equiv 1\ mod\ 65537$, 
          then 3 is a generator of $U(\mathbb{Z}/65537\mathbb{Z})$. \newline
          Therefore, 3 is a primitive root mod 65537.
\end{enumerate}

\end{document}